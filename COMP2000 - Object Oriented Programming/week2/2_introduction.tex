\documentclass[twoside=false,DIV=14]{scrartcl}

\usepackage{arev} % order matters, putting this above allows FiraSans to override it for body text
\usepackage[sfdefault]{FiraSans}
\usepackage{inconsolata}
%\usepackage[fira]{fontsetup}
\usepackage{scrlayer-scrpage}
\renewcommand{\titlepagestyle}{scrheadings}
\usepackage{graphicx}
\usepackage{blindtext}
\usepackage{wrapfig}
\usepackage{tabularx}
\usepackage{hyperref}
\usepackage{listings}
\usepackage{tikz}
\usepackage{amsmath}
\usepackage[many]{tcolorbox}

\usepackage{xcolor,sectsty}
\definecolor{blackish}{RGB}{56,58,54}
\definecolor{redish}{RGB}{109,41,49}
\definecolor{red}{RGB}{152,41,50}
\definecolor{orangeish}{RGB}{188,71,0}
\definecolor{blueish}{RGB}{25,33,139}
\subsubsectionfont{\color{blackish}}
\subsectionfont{\color{blackish}}
\sectionfont{\color{blackish}}

\lohead{\color{red} COMP3000 Programming Languages}
\rohead{\includegraphics[width=0.5cm]{../logo.jpg}}

\newcommand{\tick}{\ensuremath{\checkmark}} 
%\usepackage{enumitem}

\setkomafont{author}{\sffamily \small}
\setkomafont{date}{\sffamily \small}

\DeclareOldFontCommand{\bf}{\normalfont\bfseries}{\mathbf}
\DeclareOldFontCommand{\tt}{\normalfont\ttfamily}{\texttt}

\lstset{basicstyle=\ttfamily}


\date{}

\newtcolorbox{note}[1][]{
  title=Note,
  width=\textwidth,
  fonttitle=\bfseries,
  breakable,
  fonttitle=\bfseries\color{black},
  colframe=orangeish!80,
  colback=orangeish!2
  #1}

\newtcolorbox{hint}[1][]{
    title=Hint,
    width=\textwidth,
    fonttitle=\bfseries,
    breakable,
    fonttitle=\bfseries\color{white},
    colframe=blueish!80,
    colback=blueish!2
    #1}

\newtcolorbox{todo}[1][]{
  title=!! TODO !!,
  width=\textwidth,
  fonttitle=\bfseries,
  breakable,
  fonttitle=\bfseries\color{white},
  colframe=red!80,
  colback=red!2
  #1}
  
  

\title{\color{redish} \vspace{-1em}COMP2000 Week 2: Git}

\begin{document}
{\color{blackish}\maketitle}\vspace{-7em}

\begin{abstract}
\end{abstract}

\section*{Topics}
\begin{description}
\item[Git (GIT)]  What is a Git and how do we use it?
\end{description}
\section*{Preparation}
\begin{itemize}
\item ??
\end{itemize}


\newpage
\part*{RAT 2}
\section*{??}
??

\newpage
\part*{Application Exercise}
Your group has been tasked with making a git respository storing your work for this semester.  This week you will put into that repository:
\begin{itemize}
    \item A file containing a topic list for the whole unit.
    \item A file containing each group memeber's availability for study outside class.
    \item A file for week 1 containing a summary of the groups activities during this class. You can keep it fairly simple, you want it to be a memory aid if you ever have to explain to someone what \emph{your} group did in this class.
\end{itemize}
Ensure that the repository is on github/bitbucket/gitlab and that all group members have access to it.  You may choose to make it private, in which case you will need to deal with access permissions.

By the end of the class you should have the latest version of all these files on every member's computer.  When we re-convene, you will demonstrate making a change to one of those files and getting the change to everyone's computers.

\newpage
\part*{Self Study Exercises}
??

\end{document}