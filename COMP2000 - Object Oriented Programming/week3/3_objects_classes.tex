\documentclass[twoside=false,DIV=14]{scrartcl}

\usepackage{arev} % order matters, putting this above allows FiraSans to override it for body text
\usepackage[sfdefault]{FiraSans}
\usepackage{inconsolata}
%\usepackage[fira]{fontsetup}
\usepackage{scrlayer-scrpage}
\renewcommand{\titlepagestyle}{scrheadings}
\usepackage{graphicx}
\usepackage{blindtext}
\usepackage{wrapfig}
\usepackage{tabularx}
\usepackage{hyperref}
\usepackage{listings}
\usepackage{tikz}
\usepackage{amsmath}
\usepackage[many]{tcolorbox}

\usepackage{xcolor,sectsty}
\definecolor{blackish}{RGB}{56,58,54}
\definecolor{redish}{RGB}{109,41,49}
\definecolor{red}{RGB}{152,41,50}
\definecolor{orangeish}{RGB}{188,71,0}
\definecolor{blueish}{RGB}{25,33,139}
\subsubsectionfont{\color{blackish}}
\subsectionfont{\color{blackish}}
\sectionfont{\color{blackish}}

\lohead{\color{red} COMP3000 Programming Languages}
\rohead{\includegraphics[width=0.5cm]{../logo.jpg}}

\newcommand{\tick}{\ensuremath{\checkmark}} 
%\usepackage{enumitem}

\setkomafont{author}{\sffamily \small}
\setkomafont{date}{\sffamily \small}

\DeclareOldFontCommand{\bf}{\normalfont\bfseries}{\mathbf}
\DeclareOldFontCommand{\tt}{\normalfont\ttfamily}{\texttt}

\lstset{basicstyle=\ttfamily}


\date{}

\newtcolorbox{note}[1][]{
  title=Note,
  width=\textwidth,
  fonttitle=\bfseries,
  breakable,
  fonttitle=\bfseries\color{black},
  colframe=orangeish!80,
  colback=orangeish!2
  #1}

\newtcolorbox{hint}[1][]{
    title=Hint,
    width=\textwidth,
    fonttitle=\bfseries,
    breakable,
    fonttitle=\bfseries\color{white},
    colframe=blueish!80,
    colback=blueish!2
    #1}

\newtcolorbox{todo}[1][]{
  title=!! TODO !!,
  width=\textwidth,
  fonttitle=\bfseries,
  breakable,
  fonttitle=\bfseries\color{white},
  colframe=red!80,
  colback=red!2
  #1}
  
  

\title{\color{redish} \vspace{-1em}COMP2000 Week 3: Objects and Classes}

\begin{document}
{\color{blackish}\maketitle}\vspace{-7em}

\begin{abstract}
\end{abstract}

\section*{Topics}
\begin{description}
    \item[Objects and Classes in Java (JAV)]??
    \item[Cohereance and Coupling (CAC)] ??
\item [??] ??
\end{description}
\section*{Preparation}
\begin{itemize}
\item ??
\end{itemize}


\newpage
\part*{RAT 3}
\section*{??}
??

\newpage
\part*{Application Exercise}
Over the semester, your group will make a data visualisation/game.  In the begining, the data for you to work from will be provided in files.  Later in semester these will become streams of data from the web.  Each week you will have an activity on that application which is carefully designed to test the weeks topic.  

This week you are to get your application started.  This will require you to engage with basic Java programming and refresh your memory on how to use Java.

By the end of the hour you should be able to read in the data file you have chosen and to put up a blank Java window ready for your visualisation.

Add to your repository a file called \texttt{week 3 classes.md} which lists every single class you used and how many objects yyou created of each class.  Have that pushed to your repository and make sure everyone in your group has a copy of it.

\begin{todo}
    We need to choose one of visualisation or game.  I think visualisation will work better because we can provide a set of data streams and each group can choose a different one. We can then provide antoher one for the assignment submissions (whcih are individual work).
\end{todo}
\newpage
\part*{Self Study Exercises}
??

\end{document}