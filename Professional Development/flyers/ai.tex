\documentclass[twoside=false,DIV=14]{scrartcl}
\usepackage[sfdefault]{FiraSans}
\usepackage{scrlayer-scrpage}
\renewcommand{\titlepagestyle}{scrheadings}
\usepackage{graphicx}
\usepackage{blindtext}
\usepackage{wrapfig}


\usepackage{xcolor,sectsty}
\definecolor{blackish}{RGB}{63,64,63}
\definecolor{redish}{RGB}{67,37,52}
\definecolor{greenish}{RGB}{10,135,84}
\definecolor{purpleish}{RGB}{150,107,157}
\definecolor{orangeish}{RGB}{255,159,28}
\definecolor{brownish}{RGB}{186,168,152}
\definecolor{red}{RGB}{152,41,50}
\subsectionfont{\color{blackish}}
\sectionfont{\color{blackish}}

\rohead{\color{greenish} the ai-fluent developer}
\lohead{\color{greenish} {\huge$\eta$}-education}


\usepackage{enumitem}

\setkomafont{author}{\sffamily \small}
\setkomafont{date}{}
\date{\vspace{-5em}}
\title{\color{blackish} \vspace{-1em} The AI-Fluent Developer Course Description}

\begin{document}
{\color{blackish}\maketitle}\vspace{-2em}

\begin{abstract}
    In this course participants become better programmers by leveraging AI-tooling. Experienced developers learn how to improve development speed using AI tools \emph{without compromising on code quality or maintainability}.  AI-native developers will learn how to \emph{safely} use these tools and write code which \emph{can fit into an existing code-base}.  Participants learn how to work across the generational divide that AI has cleaved into the programming community. 
\end{abstract}

\section{For the team}
A split has developed between AI-native and more experienced developers which is hurting team productivity across the industry.  In this course both perspectives are addressed with respect and with a view to improving mutual understanding.  A key design principal of this course is to inform all developers of the old and new perspectives so that \emph{everyone} can learn from each other. If you want your team to level up its use of AI in a way that brings people along, this course is for you.

\section{For the experienced developer}
Use of AI is no longer optional for programmers.  Operating effectively in the industry means working with these tools and operating with greater velocity.  However, the tools don't operate like real programmers and bridging that gap can be challenging.  This course takes a \emph{cognitive} approach to understanding how human developers should use AI tools.  You will learn to apply the same techniques and skills you have honed over years to the latest tools.  You will also learn a few new tricks that are needed to take full advantage of them. Upon completing this course you will:
\begin{itemize}
    \item Understand how to incorporate the tools into daily work life in a constructive way.
    \item Fully understand how these tools supplement the development process from a cognitive point of view.
    \item Circumscribed and understood the limits of the tooling as they currently stand.
    \item Have seen a glimpse of the most effective ways future tooling can support quality software development on large code bases.
    \item Fully understood which programming skills have not been supplanted by AI-tooling and which have become even more important due to their presence.
\end{itemize}

\section{For the junior developer}
You may be comfortable using AI tooling but are you writing quality code which your colleagues have confidence in and can build from?  Do you really understand the code you are contributing to the codebase?   If the AI generated code turns out to be faulty, will you be equipped to fix it?  We address all these questions in this course.  Upon completing this course you will:
\begin{itemize}
    \item Understand the cognitive processes you need to bring to bear when interacting with code generators and other AI tools.
    \item Be able to incorporate AI-native quality assurance practices into your daily work.
    \item Know how to demonstrate the quality of AI-generated code more effectively than testing can.
    \item Know the limits of what AI-tooling can contribute to your work life.
    \item Level up the set of skills you need to make best use of AI-tooling.
\end{itemize}

\section{Course Design}

The course is built around \emph{the day-to-day work experience} and \emph{real code-bases}.  The course is hosted in whatever language is required by participants, the default option is to use Java.  The programming language in use \emph{does effect} the way AI is used, and thus the course is tailored for different languages.

As a team, an existing large code base will be imported into the participant's tooling of choice and we will work through the following modules:
\begin{enumerate}
    \item The breadth of tooling support on that platform.
    \item Tooling support for exploring new code.
    \item Tooling blind spots in code-bases.
    \item Modifying an existing feature of the code-base.
    \item Extending the code-base.
    \item Cognition of code and AI-tooling.
    \item Adding the human touch.
    \item Communicating code quality.
    \item Maintaining AI-generated code.
\end{enumerate}

By default, the course runs for three half-days online in a rich shared virtual environment.  The course can be run on-premises or in a hybrid mode upon request.

\section{About Us}
This course is run by Matt Roberts.  Matt has taught programming to over 10,000 people over 20 years as computer science educator.  Matt teaches at Macquarie University and the Institute of Applied Technology.  Matt has provided professional development for companies including Optus, Ghost Automotive, and Microsoft.  Matt is a former Software Engineer who has made code contributions to many open source projects including LiftWeb, Ardupilot, Moodle, Kiama Language Processor, Skink Static Analyser and has published data analysis platforms for psychological research.

\end{document}

