\documentclass[twoside=false,DIV=14]{scrartcl}

\usepackage{arev} % order matters, putting this above allows FiraSans to override it for body text
\usepackage[sfdefault]{FiraSans}
\usepackage{inconsolata}
%\usepackage[fira]{fontsetup}
\usepackage{scrlayer-scrpage}
\renewcommand{\titlepagestyle}{scrheadings}
\usepackage{graphicx}
\usepackage{blindtext}
\usepackage{wrapfig}
\usepackage{tabularx}
\usepackage{hyperref}
\usepackage{listings}
\usepackage{tikz}
\usepackage{amsmath}
\usepackage[many]{tcolorbox}

\usepackage{xcolor,sectsty}
\definecolor{blackish}{RGB}{56,58,54}
\definecolor{redish}{RGB}{109,41,49}
\definecolor{red}{RGB}{152,41,50}
\definecolor{orangeish}{RGB}{188,71,0}
\definecolor{blueish}{RGB}{25,33,139}
\subsubsectionfont{\color{blackish}}
\subsectionfont{\color{blackish}}
\sectionfont{\color{blackish}}

\lohead{\color{red} COMP3000 Programming Languages}
\rohead{\includegraphics[width=0.5cm]{../logo.jpg}}

\newcommand{\tick}{\ensuremath{\checkmark}} 
%\usepackage{enumitem}

\setkomafont{author}{\sffamily \small}
\setkomafont{date}{\sffamily \small}

\DeclareOldFontCommand{\bf}{\normalfont\bfseries}{\mathbf}
\DeclareOldFontCommand{\tt}{\normalfont\ttfamily}{\texttt}

\lstset{basicstyle=\ttfamily}


\date{}

\newtcolorbox{note}[1][]{
  title=Note,
  width=\textwidth,
  fonttitle=\bfseries,
  breakable,
  fonttitle=\bfseries\color{black},
  colframe=orangeish!80,
  colback=orangeish!2
  #1}

\newtcolorbox{hint}[1][]{
    title=Hint,
    width=\textwidth,
    fonttitle=\bfseries,
    breakable,
    fonttitle=\bfseries\color{white},
    colframe=blueish!80,
    colback=blueish!2
    #1}

\newtcolorbox{todo}[1][]{
  title=!! TODO !!,
  width=\textwidth,
  fonttitle=\bfseries,
  breakable,
  fonttitle=\bfseries\color{white},
  colframe=red!80,
  colback=red!2
  #1}
  
  

\title{\color{redish} \vspace{-1em}COMP3000 Week 1: Team Based Learning}

\begin{document}
{\color{blackish}\maketitle}\vspace{-7em}

\begin{abstract}
\end{abstract}

\section*{Topics}
\begin{description}
\item[What is TBL (TBL)]  
\item[Your rights and responsibilities (RAR)]  
\item[TBL Practice (PRT)]
\end{description}
\section*{Preparation}
\begin{itemize}
\item None this week, but in all future weeks you will need to \emph{pre-prepare} for class.  \emph{This week only} there will be a short explanation of TBL before we begin the class for real.  In future weeks \emph{there will be not lecture at the start of class}.  You will complete the RAT yourself, which is normally done \emph{before} class, then you will get into groups and complete the RAT again before we move on to the application exercise.
\end{itemize}


\newpage
\part*{RAT 1 \hspace{6em} {\small INDICATE YOUR RESPONSE TO EACH QUESTION}}
%\renewcommand{\labelenumii}{\alph{enumii}) $\fbox{$\phantom{x}$}$}
\renewcommand{\labelenumii}{\alph{enumii}) $\square$}
\begin{enumerate}
\item \textbf{Application activities should be characterized by the 4 S's. These are:}
\begin{enumerate}
    \item Substantial problem, significant choice, same problem, simultaneous report
    \item Significant problem, same problem, specific choice, simultaneous report
    \item Specific problem, same problem, significant choice, simultaneous report
    \item Significant problem, systematic problem, specific choice, simultaneous report
\end{enumerate}

\item \textbf{Which of the following outlines the steps involved when designing a TBL course in the correct chronological order?}
\begin{enumerate}
    \item Identify learning outcomes, Create application activities, Identify and/or develop preparation materials, Write Readiness Assurance Test questions, Seek feedback
    \item Identify learning outcomes, Create application activities, Write Readiness Assurance Test questions, Identify and/or develop preparation materials, Seek feedback
    \item Identify and/or develop preparation materials, Write Readiness Assurance Test questions, Create application activities, Identify learning outcomes, Seek feedback
    \item Identify learning outcomes, Identify and/or develop preparation materials, Write Readiness Assurance Test questions, Create application activities, Seek feedback
\end{enumerate}

\item \textbf{One endorsed method to overcome dysfunction within a newly formed TBL team is:}
\begin{enumerate}
    \item Well designed and difficult application tasks that facilitate rich discussion
    \item To intervene early and identify the key issues that cause the team’s dysfunction
    \item Shuffle TBL team members so that you get new members and a different team dynamic
    \item Have teams elect a team captain or team lead to run and report on discussions
\end{enumerate}

\item \textbf{The recommended method used when designing an TBL module is called:}
\begin{enumerate}
    \item Tried and true
    \item Backward design
    \item A systems approach
    \item Agile design
\end{enumerate}

\item \textbf{Which of the following is not recommended for the first class of a TBL unit?}
\begin{enumerate}
    \item Introducing students to the concept TBL and what is expected of them
    \item Conducting a graded TBL session, so students learn the hard way that they need to prepare for class
    \item Alleviating any concerns students may have on TBL, such as group grading
    \item Conducting a non-graded TBL session, so students can familiarize themselves with the TBL method in a non-threatening environment
\end{enumerate}
\end{enumerate}

\newpage
\part*{Application Exercise 1 \hspace{1em} {\small COMPLETE IN CLASS AS A GROUP}}
Your team has been assigned to develop a Team Based Learning worksheet for a topic of interest.  You should create:
\begin{enumerate}
\item a set of pre-readings, 
\item a RAT, 
\item and a set of application exercises.  
\item You should also create a set of answers for the RAT and application exercises.
\end{enumerate}

You will be given 1 hour to work on this at which point we will re-convene to discuss the various worksheets that were created.  Your group will be expected to share your worksheet with the class after 1 hour.  Your group can choose the topic of the worksheets from anything which is acceptable to \emph{polite society}\footnote{This means we avoid religion, politics and anything which could reasonably be thought to genuinely offend someone present.}
\newpage
\part*{Self Study Exercises \hspace{2em} {\small COMPLETE YOURSELF AT HOME}}
None this week, but in future weeks there will be a set of exercises, with solutions, for you to work on individually.  They will be great exam preparation.
\end{document}