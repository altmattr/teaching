\documentclass[twoside=false,DIV=14]{scrartcl}

\usepackage{dialogue}
\usepackage{arev} % order matters, putting this above allows FiraSans to override it for body text
\usepackage[sfdefault]{FiraSans}
\usepackage{inconsolata}
%\usepackage[fira]{fontsetup}
\usepackage{scrlayer-scrpage}
\renewcommand{\titlepagestyle}{scrheadings}
\usepackage{graphicx}
\usepackage{blindtext}
\usepackage{wrapfig}
\usepackage{tabularx}
\usepackage{hyperref}
\usepackage{listings}
\usepackage{tikz}
\usepackage{amsmath}
\usepackage[many]{tcolorbox}

\usepackage{xcolor,sectsty}
\definecolor{blackish}{RGB}{56,58,54}
\definecolor{redish}{RGB}{109,41,49}
\definecolor{red}{RGB}{152,41,50}
\definecolor{orangeish}{RGB}{188,71,0}
\definecolor{blueish}{RGB}{25,33,139}
\subsubsectionfont{\color{blackish}}
\subsectionfont{\color{blackish}}
\sectionfont{\color{blackish}}

\lohead{\color{red} COMP3000 Programming Languages}
\rohead{\includegraphics[width=0.5cm]{../logo.jpg}}

\newcommand{\tick}{\ensuremath{\checkmark}} 
%\usepackage{enumitem}

\setkomafont{author}{\sffamily \small}
\setkomafont{date}{\sffamily \small}

\DeclareOldFontCommand{\bf}{\normalfont\bfseries}{\mathbf}
\DeclareOldFontCommand{\tt}{\normalfont\ttfamily}{\texttt}

\lstset{basicstyle=\ttfamily}


\date{}

\newtcolorbox{note}[1][]{
  title=Note,
  width=\textwidth,
  fonttitle=\bfseries,
  breakable,
  fonttitle=\bfseries\color{black},
  colframe=orangeish!80,
  colback=orangeish!2
  #1}

\newtcolorbox{hint}[1][]{
    title=Hint,
    width=\textwidth,
    fonttitle=\bfseries,
    breakable,
    fonttitle=\bfseries\color{white},
    colframe=blueish!80,
    colback=blueish!2
    #1}

\newtcolorbox{todo}[1][]{
  title=!! TODO !!,
  width=\textwidth,
  fonttitle=\bfseries,
  breakable,
  fonttitle=\bfseries\color{white},
  colframe=red!80,
  colback=red!2
  #1}
  
  


\title{\color{redish} \vspace{-2em}Week 3 Workshop: Correctness and performance of programs}

\begin{document}
{\color{blackish}\maketitle}\vspace{-2em}


\begin{itemize}
\item[$\cdot$] {\bf Resources}: Week 1 and 2 code bundles.
\item[$\cdot$] {\bf To submit this week's work:} Submit your solution to Exercise \ref{sec:submission} to your teacher in class.  You should hand in a single piece of paper with your solution on it.  Be sure you include your name, student number, and the week number in the top-right of your submission.
\end{itemize}


\part*{Exercises}

\section{Problem Solving: Warmup}
  How much does the ice in a hockey rink weigh? 
    
  For this question, getting exactly the right answer is not the point (although it's good when it's in the ``ball park").
  Questions like these help you to practise thinking about a problem and to define the assumptions that need to be
  made to give a reasonable answer. In your group, discuss some assumptions you can make to enable you
  to do a ``back of the envelope" calculation. In your solution list the assumptions that you made in order to
  come up with your answer.
  
  Note also that in ``real life" most problems are ill defined (such as this one) and to be a good problem
  solver and critical thinker, you need to be able to formulate the problem in a way that gives something of
  a meaningful (if not always useful!) answer.
  
\section{Invariants: Basics - Submission}
\label{sec:submission}

Consider the following program code. Assume that there is a given linked list with first node {\tt head}, and a variable {\tt x} initialised to some value.

\begin{lstlisting}
char x = <<?>>; LinkedList head = <<?>>;
LinkedList temp = head;
while(temp != null && temp.value != x) {
   temp= temp.next;
}
\end{lstlisting}

For each of the example lists in the diagram (each containing 4 nodes) diagrammatically show which object/node {\tt temp} points to after running this algorithm.

\usetikzlibrary {graphs}
\vspace{1em}
\begin{tabular}{lr}
\hline\\
\verb|char x = 'b';| & \tikz [rotate=90, var/.style={draw=none}, baseline, font=\ttfamily]\graph [nodes={rounded corners,draw, minimum height=2em},edges={nodes={anchor=mid}},grow right=4em]{head[var] -> {'a'} -> {'b'} -> {'c'} -> {'d'} -> phi/$\phi$[var],}; \\
\\ \hline \\ 
\verb|char x = 'b';| & \tikz [rotate=90, var/.style={draw=none}, baseline, font=\ttfamily]\graph [nodes={rounded corners,draw, minimum height=2em},edges={nodes={anchor=mid}},grow right=4em]{head[var] -> {'a'} -> yep[as={'a'}] -> {'b'} -> bb[as={'b'}] -> phi/$\phi$[var],}; \\
\\ \hline \\
\verb|char x = 'd';| & \tikz [rotate=90, var/.style={draw=none}, baseline, font=\ttfamily]\graph [nodes={rounded corners,draw, minimum height=2em},edges={nodes={anchor=mid}},grow right=4em]{head[var] -> {'a'} -> {'b'} -> {'c'} -> bb[as={'b'}] -> phi/$\phi$[var],}; \\
\end{tabular}

\section{Invariants: Any old thing}
Predicates belong between two lines of code.  Create six predicates for each possible slot (before line one, between line one and two, ...., between line four and five, after line five) of the code in question \ref{sec:submission}.

Don't worry about how useful your predicates are, just that they are true.

\section{Invariants: Simple Program}

Predicates are "Invariants" if they are true in every slot of the program.  Consider the following invariant of the program above:

\begin{quotation}
For all nodes {\tt n} lying between {\tt head} and {\tt temp} (but not including {\tt temp}) it is the case that {\tt n.value != x}.
\end{quotation}


In the case that {\tt temp} is {\tt null} because it has traversed the whole list, then take this statement to mean ``all nodes in the list''.

\begin{enumerate}
\item  If the loop ends with {\tt temp == null}, what does the invariant tell us about the contents of the list?

\item  If the loop ends with {\tt temp != null}, what does the invariant tell us about the contents of the list?
%\end{enumerate}
\item Hence or otherwise, suggest a suitable post-condition for this loop to tell you what it does.
\end{enumerate}

\section{Loop Invariants: Post-conditions}
 
Consider the variations below of the sum algorithm discussed in lectures. In each case use the loop invariant given and work out the post-condition by putting the invariant together with the "negation of the guard". Once you have the post-condition, put into words what the loops do. {\bf You may assume that the loop invariant given has been checked and that it is indeed and loop invariant.}
 
 \begin{enumerate}
 \item 
 \begin{verbatim}
int i=0;  int sum=0;
while ( i <= N ) {
   i= i+1;
  sum= sum + i;
}
\end{verbatim}
 Use loop invariant $\texttt{sum} = \sum_{j=0}^{j=i} j$
 
  \item 
 \begin{verbatim}
int i=1;  int sum=0;
while ( i < N ) {
  sum= sum + i;
   i= i+1;
}
\end{verbatim}
 Use loop invariant $\sum_{j=0}^{j=i-1} j ={\tt sum}$
 
   \item 
 \begin{verbatim}
int i=1;  int sum=0;
while ( i < N+1 ) {
  sum= sum + i;
   i= i+1;
}
\end{verbatim}
 Use loop invariant $\sum_{j=0}^{j=i-1} j ={\tt sum}$
 
 \end{enumerate} 

\section{Loop Invariants: Program Correctness}  
Consider the following small program:
 
 \begin{verbatim}      
//  pre  :  X <= Y
//  post : returns the sum X + (X+1) + ... + Y 
int sumBetween (int X, int Y){
  int sum = 0;
  for (int i=X; i!=Y+1; i++)
      sum += i;
  return sum;
}
\end{verbatim}
    
\begin{enumerate}
\item    State the loop invariant. (Hint: Use the example for summing the numbers between 0 and N  discussed in lectures as a starting point.)
 \item   Show that the loop invariant is established by the initialization.
  \item  Verify that the loop invariant is preserved by each iteration of the loop.
  \item  Show that the loop invariant and the termination condition imply the post-condition. 
  \item What is the complexity in big oh notation of this program?
\end{enumerate}

\section{Performance: Big Oh}
Recall that Big Oh is a \emph{shorthand notation} for the rate of growth based in input size. For any program, you can work out a relationship between the input size ($n$) and the run-time.  This relationship is then given as a function.  If a program gets 1 cycle slower every time one new element goes into the input, it's performance function is $n$.  If a program gets 10 cycles slower for each new input, its performance function is $10n$

Recall also that Big Oh is a \emph{rough} measure, which captures only the \emph{shape} of the function.  Both the $n$ and $10n$ performance functions are straight lines and thus should have the same Big Oh value, in this case $O(n)$.  

Work out the big Oh equivalent of the following performance functions.

\begin{enumerate}
\item $n^2 + 100n$
\item $n\times (4n + 20)$

\item $400$

\item $n^2 + \log n + 11$
\end{enumerate}

\begin{hint}
Here is the table from lectures to help

{\scriptsize
\begin{tabular}{llll}
    \hline 
    \textbf{description} & \textbf{iterative} & \textbf{recursive} & \textbf{Big Oh} \\
    \hline
    No loop & $T(n) = c_1$ &  & $O(1)$ \\
    Normal loop & $T(n) = c_1 + c_2*n$  & $T(n) = C_1 + C_2*T(n-1)$ & $O(n)$ \\
    Doubled Halving & & $T(n) = T(n/2) + T(n/2)$ & $O(n) $ \\
    Nested loop & $T(n) = n*n$ & $T(n) = c_1 + c_2*T(n-1)*T(n-1)$ & $O(n^2)$ \\
    Halving loop & $T(n) = n + n/2 + n/4 + \ldots + 1$ & $T(n) = C_1 + T(n/2) + T(n)$ & $O(\log{n})$ \\
    Work on both half & $T(n) = 2*(n + n/2 + \ldots + 1)$ & $T(n) = C_1 + T(n) + 2*T(n/2)$ & $O(n \log{n})$ \\
    Recursion on two  & & $T(n,m) = T(n-1, m) + T(n, m-1)$ & $O(2^{max(n,m)})$ \\
    Fill in a 2d table & $T(n) = n*m$ & $T(n,m) = n*T(m)$ & $O(n \times m)$ \\
\end{tabular}
}
  
\end{hint}


       

\newpage\setcounter{section}{0}
\part*{Exercise Solutions}
\section{Problem Solving: Warmup}

No set solution for this one, see what the students come up with.

\section{Invariants: Basics - Submission}

\includegraphics[width=\textwidth]{./week3_soln1.jpeg}

\section{Invariants: Any old thing}
Some of these will relate to the code, some won't, but they will all be true for that slot of the program!

\vspace{1em}
\begin{tabular}{l r}
\hspace{4em}$True$  & this is a silly one \\
\verb|char x = <<?>>; LinkedList head = <?>;| \\
\hspace{4em}$True$ & also silly, but definitely true! \\
\verb|LinkedList temp = head;| \\
\hspace{4em}$temp = \phi \lor temp \neq \phi$ & I don't know, so this is all I can say! \\
\verb|while(temp != null && temp.value != x) {| \\
\hspace{4em}$temp \neq \phi$ & In here I know it is not null though \\
\verb|  temp= temp.next;| \\
\hspace{4em}$temp = \phi \lor temp \neq \phi$ & but now I am back to not knowing \\
\verb|}| \\
\hspace{4em}$True$ & another silly one \\
\end{tabular}
  

\section{Invariants: Simple Program}

So, I have this invariant
\begin{quotation}
  For all nodes {\tt n} lying between {\tt head} and {\tt temp} (but not including {\tt temp}) it is the case that {\tt n.value != x}.
  \end{quotation}

\begin{enumerate}
\item It tells us that the list didn't contain the character we were looking for.
\item It tells us the character was in the list.
\item Since this loop finds an element in a list, we would like the postcondition to include this information.  Postconditions only have to be true a the end of the loop/function so we only need to worry about that last slot.  How about "temp either points to the first instance of x in the list, or it points to null"?  That's true at the end of the loop and says what we want.  Another way to say might be "no elements between head and temp contain x and temp either contains x or is null".
\end{enumerate}

\section{Invariants: Postcondition}
This question is trying to help you see the \emph{utility} of invariants.  If you do something as simple and mechanical as combine the invariants with the negation of the termination condition you can extract a precise definition of the meaning of the program - I think that's pretty amazing.

However, this problem is one that highlights what is wrong with "negation of termination condition" as the standard terminology - what is the negation of $ i <= N$?  It is $i > N$ but that is useless for this question. The question actually needs $i = N + 1$ for this to work. So the scheme of combining the loop invariant with the negation of the guard is useless.  I actually have to combine it with a more specific statement about the end of the loop.

Recall that a postcondition is anything true at the end of the loop/function.  A \emph{useful} postcondition is one that is true at that point \emph{and} tells us something of value.  One strategy for finding a postcondition is to find a loop \emph{invariant} and combine it with the condition that becomes true at the end of the loop - the negation of the guard\footnote{it is not \emph{actually} the negation of the guard.  It is more like "a statement that is true when the guard first becomes false" but the textbooks all use this incorrect term, as do the notes, so who am I to change it?}!  We do that here.
\begin{enumerate}
\item The negation of the guard is $i > N$ but the more precise and useful predicate is $i = N + 1$.  You should always be looking for a predicate with equality because then you can substitute it into the invariant.  Thus, combine invariant ${\tt sum = \sum_{j=0}^{j=i} j}$ with $i = N+1$ giving ${\tt sum} = \sum_{j=0}^{j=N+1} j $.  Thus we can say this code computes the sum of the numbers from $0$ to $N+1$.
\item Combine  invariant ${\tt sum = \sum_{j=0}^{j=i-1} j}$ with $i = N$. This gives ${\tt sum} = \sum_{j=0}^{j=N-1} j $.  Thus we can say this code computes the sum of the numbers from $0$ to $N-1$.
\item Combine  invariant ${\tt sum = \sum_{j=0}^{j=i-1} j}$ with $i = N + 1$. This gives ${\tt sum} = \sum_{j=0}^{j=N} j $.  Thus we can say this code computes the sum of the numbers from $0$ to $N$.
\end{enumerate}

\section{Invariants: Program Correctness}
Recall that we can use loop invariants in the following way to show the correctness of code:
\begin{enumerate}
\item Get your pre and post conditions.  These are the things true before your code snippet (pre-condition) and the thing you want to be true after your code snippet (post-condition).  These often are given to you.
\item Choose a loop invariant that looks useful to you.  It is useful if you can combine it with the "negation of the guard" to get your post-condition.
\item Prove the correctness of your invariant, which you do by:
\begin{enumerate}
\item showing the loop initialiser make the invariant true
\item show that \emph{if} the loop invariant is true at the start of the loop, \emph{then} it is true at the end of the loop
\end{enumerate}
\item Prove the \emph{value} of your loop invariant, which you do by showing that \emph{the pre condition} and \emph{the loop invariant} and \emph{the "negation of the termination condition"} give the post condition.
\end{enumerate}
And you are all done.

Our post-condition is $sum = \sum_{j = X}^{j=Y}j$ and we know that at the end of the loop $i = Y + 1$.  This gives me a major clue, given what I've seen before.  I am going to try this loop invariant
$$
sum = \sum_{j = X}^{j = i -1}j
$$

\begin{note}
I am writing this solution in quite a mathematical way.  Recall we have a spectrum of proof roughly like: feel $\rightarrow$ demonstration $\rightarrow$ argument $\rightarrow$ structured argument $\rightarrow$ logical argument $\rightarrow$ proof $\rightarrow$ rigorous proof.

You only need to reach \emph{logical argument} but here I am straying close to \emph{proof}
\end{note}
\subsection{Showing that the initialisers make the invariant true}
After initialisers:
\begin{itemize}
\item $sum = 0$
\item $i = X$
\end{itemize}
Thus the invariant is
$$
0 = \sum_{j = X}^{j = X-1}j
$$
A sum where the top limit is lower than the bottom limit is strange, but fine and certainly equals 0.

\subsection{Showing that the loop maintains the invariant}
NB: When doing \verb|for| loops, you need to act as if the increment is in the loop body.

NB2: I am giving a full treatment here, but in this course you only need to explain in a convincing manner.  I'm just so deep in maths, that the only "convicing manner" I know.

A trick I use to help is to think of the variables involved having two states, the start state ($\alpha$) and end state ($\zeta$).  They get updated during the loop, so I need to keep track of both values.

Our assumptions are:
\begin{flalign}
sum_\alpha= \sum_{j = X}^{j = i_\alpha -1}j & \hspace{3em} &\textrm{true at start/assumption} \\
sum_\alpha + i = sum_\zeta & & \texttt{sum+=i} \\
i_\alpha + 1 = i_\zeta  & &  \texttt{i++}
\end{flalign}
We are trying to prove
\begin{flalign}
sum_\zeta = \sum_{j = X}^{j = i_\zeta -1}j & & \textrm{post-condition}
\end{flalign}
Substituting (2) and (3) into (4) gives
\begin{flalign}
  sum_\alpha + i = \sum_{j = X}^{j = i_\alpha +1 -1}j & & 
\end{flalign}
Using (1) to substitute for $sum_\alpha$ and simplifying the upper limit of the sum gives
\begin{flalign}
  \sum_{j = X}^{j = i_\alpha -1}j + i = \sum_{j = X}^{j = i_\alpha}j & & 
\end{flalign}
Which is definately true!  (6) is just saying that the sum one less than $i$ plus $i$ is the same as the sum to $i$ which sounds good to me.

So we have done both steps of proving we got a valid loop invariant.

The final thing the question asked me to do is show the value of my invariant, i.e...

\subsection{Show the loop invariant and termination condition give the post condition}

Combining the invariant $sum = \sum_{j = X}^{j = i -1}j$ with a \emph{useful} termination condition $i = Y + 1$ gives 

$$
sum = \sum_{j = X}^{j = y}j
$$
Thus we can say this loop sums the numbers from $X$ to $Y$, just as we hoped.

\subsection{Big-Oh}
The Big Oh of this program is $O(Y-X)$.  You are probably used to just having a single letter in the Oh, but what goes in there is the input size term, even if that is more complicated.  In this case that is $Y-X$ and the function is linear in this term.

\section{Performance: Big Oh}
In all cases, we expand out the term and look for the largest growth factor.  We then drop all other growth factors and any mutliplicative elements leaving just the largest growth factor.
\begin{align*}
n^2 + 100n  & \hspace{3em} & O(n^2) \\
n\times (4n + 20) & &  O(n^2) \\
400 & & O(1) \\
n^2 + \log n + 11  & & O(n^2)
\end{align*}

\newpage\setcounter{section}{0}
\part*{Self Study Exercises}

\section{Invariants: Insertion Sort} Consider the implementation below for Insertion Sort, which sorts an array into descending order.
\begin{enumerate}
\item Trace the algorithm for the array {\tt A = \{2,1,5,0,6\}}, and {\tt n=5}; make sure you keep track of the array after each inner and outer iteration. In your trace, draw a rectangle around the array after each complete iteration of the OUTER loop (see (++) in the code). 

\item Verify that the array portion {\tt A[0..j-1]} is sorted after each complete iteration of the outer loop, i.e. underline that portion of the array and check that it is sorted.  (Note that {\tt A[0..j-1]} means ``the portion of the array between, and including, indices {\tt 0} and {\tt j-1}.) 


\item Now check that the invariant for the outer loop is established (made to be true) by the initialisation of the variable j; i.e. substitute the initial values of the relevant variables into the invariant statement and observe that it is true.

\item Finally deduce that the postcondition is satisfied after the outer loop has terminated. 

[Hint: what is the value of {\tt j} on termination? Substitute into the invariant.]

\end{enumerate}

\begin{verbatim}
// PRE: array.length >= 0
// POST: array is sorted
void insertion_sort(int array[]) {
     int i, j, key;
     for(j = 1; j < array.length; j++) {    //Notice starting with 1 (not 0)
     // Invariant for the outer loop: array[0..j-1] is sorted, descending order
           key = array[j];
           for(i = j - 1; (i >= 0) && (array[i] < key); i--){
              //Move smaller values up one position
              array[i+1] = array[i];
           }
           array[i+1] = key;    //Insert key into proper position (++)
       }
      return;
}
\end{verbatim}  

\section{Performance: Big Oh}
Work out the big Oh equivalent of the following performance functions.

\begin{enumerate}
\item $n^2 + n\log n$
\item $n\times (4n + \log n)$

\item $4 \times 10^126$

\item $n^2 * \log n + 11$
\end{enumerate}

(\textit{Hint:} Your answer should look like one of the following:
$O(1)$, $O(\log n)$, $O(n)$, $O(n \log n)$, $O(n^2)$, $O(n^2 \log n)$, $O(n^3)$, $O(n^3 \log n)$, or $O(2^n)$.%
)

\section{Invariants: Recursive Function}
Consider the following function on linked lists (use the definition of \lstinline|LinkedList| discussed in class and in the code bundle).

\begin{lstlisting}[language=java]
LinkedList splitAt(char c){
  if (next == null){
      return null;
  } else {
      if (value == c){
          LinkedList ret = next;
          next = null;
          return ret;
      } else {
          return next.splitAt(c);
      }
  }
}
\end{lstlisting}
Come up with a post-condition for this function.  You know what it is trying to do, so come up with something \emph{precise} that expresses that and relates to variables in the program.

\section{Invariants: Iterative Function}
Consider the following function on linked lists (use the definition of \lstinline|LinkedList| discussed in class and in the code bundle).

\begin{lstlisting}[language=java]
  LinkedList splitAt_iterative(char c){
    LinkedList curr = this;
    while(curr != null){
        if (curr.value == c){
            LinkedList ret = curr.next;
            curr.next = null;
            return ret;
        }
        curr = curr.next;
    }
    return null;
  }
\end{lstlisting}
Come up with a post-condition for this function.  You know what it is trying to do, so come up with something \emph{precise} that expresses that and relates to variables in the program.  Note that this is the same function as above, but written in an iterative style

% \section{Invariants: Revealing an Error}
% \begin{todo}
% I think this is a very interesting problem, worth working out.  Its a variant on binary search which makes things fun
% \end{todo}

% Here is an implementation of Binary Search that unfortunately a friend did in a hurry and so didn't have time to test it. It does assume that the input array A is sorted and that the key is an item which might (or might not) be an item occurring in it.

% \begin{enumerate}
% \item In the case that key does appear in A, the integer returned should be the smallest index such that A[index] == key.

% \item In the case that key does not appear in A then the smallest index should be returned such that everything to the left of index is strictly less than key, and everything to the right is at least key. (Note this is a neat way of saying the same thing as 1 in the case that key is in A ....)
% \end{enumerate}


% Use a loop invariant to work out if this code is correct or not.
% \begin{hint}
% Use the loop invariant $A[f-1] < key \land A[l] >= key$. i.e. "the key is greater than \verb|A[f-1]| and less than or equal to \verb|A[l]|.
% \end{hint}

% \begin{verbatim}
% // PRE: A is in ascending order
% static int binarySearch(char A[], char key) {
%   int f= 0;
%   int l= A.length;
   
%   while(f < l) {
%     int mid= (l-f)/2;
%     int x= A[mid];
%     if (x == key) return mid;
%     if (x < key) f= mid;
%     else l= mid;
%   }
 
% return f;
% }
% \end{verbatim}

\section{Invariants: Basic Concepts}
Have a conversation with your preferred Generative AI tool to find out if it can accurately explain \emph{how can you use loop invariants to prove the correctness of a program snippet}.  Try and find an error in it's responses if you can.

\newpage\setcounter{section}{0}
\part*{Self Study Solutions}
\section{Invariants: Insertion Sort}

Here is an image from the end of my trace, notice the locations $++$.  \href{https://youtu.be/c_kWWG8UKhQ?si=r34dl8X6jzAZorpK}{Watch this vid for the full trace\footnote{\url{https://youtu.be/c_kWWG8UKhQ?si=r34dl8X6jzAZorpK}}}.

\includegraphics[width=0.7\textwidth]{./insertion_trace.jpeg}

\begin{enumerate}
\item none
\end{enumerate}

\section{Performance: Big Oh}
\begin{enumerate}
\item $O(n^2)$
\item $O(n^2)$

\item $O(1)$

\item $O(n^2\log n)$
\end{enumerate}

\section{Invariants: Recursive Function}
It can be difficult to work out an invariant, recall our strategies:
\begin{itemize}
\item Consider the state you want all the variables to be in when the program is working correctly.
\item Allow yourself some freedom in how you express that idea.
\item Think about all possible states.
\end{itemize}
Splitting is all about whether the next variable will point to where it used to, or be adjusted to point at null.  This gives me the start
\begin{quote}
\lstinline|next == null| or \lstinline|next != null|
\end{quote}
Clearly not helpful yet, but I know that \emph{if} the current value is \lstinline|c| then I definitely want next to be null
\begin{quote}
  if \lstinline|value == c| then \lstinline|next == null|
\end{quote}
We would need to do some further work \emph{using} this post-condition to work out if it is useful, but I think it is a good start.  Note that you can't get all that much into this post-condition and this is a common situation in recursive function definitions.  However, the fact that the definition is recursive means that we can have the same post condition for every node in the list and \emph{that} is often how we make these fairly weak post conditions actually useful.

\section{Invariants: Iterative Function}
You should notice that we have an extra variable to work with here.  Since our strategies are built around variables, we might have more to work with here.  Let's see.  We can start from the same place
\begin{quote}
if \lstinline|value == c| then \lstinline|next == null|
\end{quote}
but because we have the variables \lstinline|curr| and \lstinline|this| variables which are related to each other, we can add something!
\begin{quote}
(if \lstinline|value == c| then \lstinline|next == null|) and (no node between \lstinline|this| and \lstinline|curr| has value of \lstinline|c|)
\end{quote}
This feels better and is more immediately useful.  This little exercise might help you see why we tend to set invariant tasks on iterative code rather than recursive code.

\begin{note}
When you really get into it, the recursive algorithms \emph{are} easier to prove things about, but you need to be further down the track to see that.  If you are interested in this, aim to do COMP7010 in your Masters degree.
\end{note}

% \section{Loop Invariants: Revealing an Error}
% \begin{hint}
%   Use the loop invariant $A[f-1] < key \land A[l] >= key$. i.e. "the key is greater than \verb|A[f-1]| and less than or equal to \verb|A[l]|.
%   \end{hint}
  
%   \begin{verbatim}
%   // PRE: A is in ascending order
%   static int binarySearch(char A[], char key) {
%     int f= 0;
%     int l= A.length;
     
%     while(f < l) {
%       int mid= (l-f)/2;
%       int x= A[mid];
%       if (x == key) return mid;
%       if (x < key) f= mid;
%       else l= mid;
%     }
   
%   return f;
%   }
%   \end{verbatim}

Recall that the process for using a loop invariant is:
\begin{enumerate}
  \item Pick a loop invariant.  Ours has been chosen as $A[f-1] < key \land A[l] >= key$.
  \item Prove the loop invariant is \emph{indeed} a loop invariant.
  \item Show that the loop invariant combined with pre-conditions and the loop termination will give the post condition.
\end{enumerate}
Let's start by proving the loop invariant is \emph{indeed} a loop invariant.
\subsection{Checking loop invariant}
This requires two steps:
\begin{enumerate}
\item The invariant is true at the top of the first iteration of the loop
\item Knowing the invariant is true at the start plus the effects of the code give the loop invariant being true at the end of the loop.
\end{enumerate}
\subsubsection{On the first iteration}
$f = 0$, $l = end$.  It is possible our loop invariant is \emph{not} satisfied.  If $key$ is entirely below or entirely above the array, then this will fail.

\section{Invariants: Basic Concepts}
I used Microsoft CoPilot and it overall \emph{seemed} accurate.  It made two errors:
\begin{enumerate}
\item It identified \emph{the invariant maintenance} step as the key to getting the right termination condition, which is wrong.  It is quite a substantial error which I think puts the answer overall into the category of "incorrect".  Only someone with good strong knowledge would find the error.
\item I made a false statement at the end and it didn't challenge me.  It said "Exactly" when I was actually wrong.  It \emph{always} starts by telling you that you are right, even if you are not.
\end{enumerate}
Thus, I think CoPilot is not a good study aid on this topic.  Crucially, 99\% of what it says is clear and understandable but there is a fundamental error in there that will prevent anyone really understanding the truth.  Here is my full conversation
\begin{dialogue}
\speak{Matt} Tell me how a loop invariant can be used to prove the correctness of a program snippet.
\speak{CoPilot} A loop invariant is a condition that holds true before and after each iteration of a loop. It can be used to prove the correctness of a program snippet by following these steps:
\begin{description}
  \item[Initialization] Prove that the invariant holds before the loop starts. This ensures that the invariant is true at the beginning of the first iteration.

  \item[Maintenance]: Prove that if the invariant holds before an iteration of the loop, it remains true after the iteration. This step ensures that the invariant is maintained throughout all iterations of the loop.

  \item[Termination]: Prove that when the loop terminates, the invariant, combined with the loop's termination condition, implies the correctness of the program. This step ensures that the desired post-condition of the loop is achieved when the loop exits.
\end{description}
Let's consider a simple example to illustrate this:

\begin{lstlisting}[language=python]
# Program snippet to find the sum of the first n natural numbers
def sum_natural_numbers(n):
    sum = 0
    i = 1
    while i <= n:
        sum += i
        i += 1
    return sum
\end{lstlisting}

\begin{description}
  \item[Loop Invariant]: At the start of each iteration of the loop, \verb|sum| is the sum of the first \verb|i-1| natural numbers.

  \item[Initialization]: Before the loop starts, \verb|i = 1| and \verb|sum = 0|. The sum of the first \verb|0| natural numbers is \verb|0|, so the invariant holds.

  \item[Maintenance]: Assume the invariant holds at the start of an iteration, i.e., \verb|sum| is the sum of the first \verb|i-1| natural numbers. During the iteration, \verb|i| is added to \verb|sum|, and then \verb|i| is incremented by \verb|1|. After the iteration, \verb|sum| is the sum of the first \verb|i| natural numbers, and \verb|i| is incremented to \verb|i+1|. Thus, the invariant holds for the next iteration.

  \item[Termination]: The loop terminates when \verb|i > n|. At this point, \verb|i = n+1|, and \verb|sum| is the sum of the first \verb|n| natural numbers. Therefore, the post-condition is satisfied, and the program is correct.
\end{description}
By proving these three steps, we can confidently say that the program snippet correctly computes the sum of the first \verb|n| natural numbers.
\speak{Matt} How do you know from the termination condition that i = n + 1 and not (say) i = n + 5, which also satisfies the termination condition?
\speak{CoPilot} Great question! The key lies in the \emph{maintenance} step of the loop invariant proof.

In the given program snippet, the loop invariant is that \verb|sum| is the sum of the first \verb|i-1| natural numbers at the start of each iteration. During each iteration, \verb|i| is incremented by exactly \verb|1| (\verb|i += 1|). This means that \verb|i| increases sequentially by 1 in each iteration.

Let's break it down:

\begin{description}
  \item[Initialization]: Before the loop starts, \verb|i = 1|.
  \item[First Iteration]: \verb|i| is incremented to \verb|2|.
  \item[Second Iteration]: \verb|i| is incremented to \verb|3|.
  \item[...]: This pattern continues until \verb|i| reaches \verb|n|.
\end{description}

When the loop terminates, the condition \verb|i <= n| is no longer true. This means \verb|i| must be greater than \verb|n|. Since \verb|i| is incremented by \verb|1| in each iteration, the smallest value \verb|i| can have that is greater than \verb|n| is \verb|n + 1|.

If \verb|i| were to be \verb|n + 5|, it would imply that \verb|i| was incremented by more than \verb|1| in some iteration, which contradicts the loop's increment operation (\verb|i += 1|). Therefore, the only value that satisfies the termination condition and maintains the loop invariant is \verb|i = n + 1|.

Does that help clarify things?
\speak{Matt} I think you are right in the end, but I don't think it is the maintenance step that is the key to that. Maintenance shows the invariant holds but says nothing about how the termination condition progresses.  You've used some other information to show that.
\speak{CoPilot} You're absolutely right. The maintenance step ensures that the loop invariant holds true throughout the iterations, but it doesn't directly address how the termination condition progresses.

The key to understanding why \( i = n + 1 \) at termination lies in the \emph{increment operation} within the loop and the \emph{termination condition} itself. Here's a more precise breakdown:
\begin{description}
  \item[Initialization]: Before the loop starts, \( i = 1 \).
  \item[Increment Operation]: In each iteration, \( i \) is incremented by exactly 1 (\( i += 1 \)). This means \( i \) progresses sequentially: 1, 2, 3, ..., up to \( n \).
  \item[Termination Condition]: The loop continues as long as \( i \leq n \). When \( i \) exceeds \( n \), the loop terminates.
\end{description}
Given these points, when the loop terminates, \( i \) must be the smallest integer greater than \( n \). Since \( i \) is incremented by 1 in each iteration, the smallest value \( i \) can have that is greater than \( n \) is \( n + 1 \).

So, the combination of the increment operation and the termination condition ensures that \( i = n + 1 \) when the loop exits.

Thanks for pointing that out! Does this explanation make it clearer?
\speak{Matt} Yes, thanks. However, I can start at a number greater than n and then the loop might never terminate, right?
\speak{CoPilot} Exactly! If the initial value of \( i \) is greater than \( n \), the loop's termination condition \( i \leq n \) would be false right from the start, and the loop would never execute. This is an important aspect to consider when proving the correctness of a loop.

To ensure the loop behaves as expected, it's crucial to verify that the initial conditions are set correctly. In our example, the initial value of \( i \) must be \( 1 \) (or any value less than or equal to \( n \)) for the loop to function properly and eventually terminate.

Would you like to explore more examples or dive deeper into another topic?
\end{dialogue}
\end{document}
