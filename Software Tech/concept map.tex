% !TeX TS-program = lualatex
\documentclass[landscape,twoside=false,DIV=14]{scrartcl}
\usepackage{multirow}

\usepackage{arev} % order matters, putting this above allows FiraSans to override it for body text
\usepackage[sfdefault]{FiraSans}
\usepackage{inconsolata}
%\usepackage[fira]{fontsetup}
\usepackage{scrlayer-scrpage}
\renewcommand{\titlepagestyle}{scrheadings}
\usepackage{graphicx}
\usepackage{blindtext}
\usepackage{wrapfig}
\usepackage{tabularx}
\usepackage{hyperref}
\usepackage{listings}
\usepackage{tikz}
\usepackage{amsmath}
\usepackage[many]{tcolorbox}

\usepackage{xcolor,sectsty}
\definecolor{blackish}{RGB}{56,58,54}
\definecolor{redish}{RGB}{109,41,49}
\definecolor{red}{RGB}{152,41,50}
\definecolor{orangeish}{RGB}{188,71,0}
\definecolor{blueish}{RGB}{25,33,139}
\subsubsectionfont{\color{blackish}}
\subsectionfont{\color{blackish}}
\sectionfont{\color{blackish}}

\lohead{\color{red} COMP3000 Programming Languages}
\rohead{\includegraphics[width=0.5cm]{../logo.jpg}}

\newcommand{\tick}{\ensuremath{\checkmark}} 
%\usepackage{enumitem}

\setkomafont{author}{\sffamily \small}
\setkomafont{date}{\sffamily \small}

\DeclareOldFontCommand{\bf}{\normalfont\bfseries}{\mathbf}
\DeclareOldFontCommand{\tt}{\normalfont\ttfamily}{\texttt}

\lstset{basicstyle=\ttfamily}


\date{}

\newtcolorbox{note}[1][]{
  title=Note,
  width=\textwidth,
  fonttitle=\bfseries,
  breakable,
  fonttitle=\bfseries\color{black},
  colframe=orangeish!80,
  colback=orangeish!2
  #1}

\newtcolorbox{hint}[1][]{
    title=Hint,
    width=\textwidth,
    fonttitle=\bfseries,
    breakable,
    fonttitle=\bfseries\color{white},
    colframe=blueish!80,
    colback=blueish!2
    #1}

\newtcolorbox{todo}[1][]{
  title=!! TODO !!,
  width=\textwidth,
  fonttitle=\bfseries,
  breakable,
  fonttitle=\bfseries\color{white},
  colframe=red!80,
  colback=red!2
  #1}
  
  

\KOMAoptions{paper=a3}
\renewcommand\pagemark{}  % remove page numbers
\usepackage{tikz}
\usetikzlibrary {graphs,graphdrawing} \usegdlibrary {trees,force,layered} 

\begin{document}
\section*{First Year}
\tikz \graph [sweep crossing minimization,minimum height layer assignment,layered layout, nodes={draw,rounded corners}, level distance=15mm,grow=up] {
  References and Objects [olive] -> Objects and Classes [olive] -> this keyword [olive] -> Constructors and new keyword[olive] -> Static Methods [olive] -> Object Lifecycle [olive] -> Object Composition [olive] ->  Java Proficiency [olive],
  Sequencing [red] -> Variables [red] -> Conditions [red] -> Loops [red] -> Functions [red],
  Functions -> Overloading [olive],
  Functions -> Values vs Expressions [red],
  Algebra [green] -> Values vs Expressions -> new Keyword [red] -> Reference Semantics [red] -> Structured Data [red] -> References and Objects [red],
  Boolean Logic -> Conditions,
  Reference Semantics [red] -> Arrays [red] -> Array Manipulation [olive] -> Searching and Sorting [olive],
  Searching and Sorting [olive] -> Algorithm Comparisons [olive],
  Constructors and new keyword -> The Object Class [olive] -> Java Proficiency [olive],
  Overloading -> Constructors and new keyword, % wish I could add this but layout is not dealing with it well
  Boolean Logic -> First Order Logic -> Quantifiers -> Math Literacy,
  Math Functions -> Recurrence Relations -> Math Literacy,
  Math Sequences -> Math Literacy,
  File Systems [green] -> Sequencing,
  Algebra -> Boolean Logic,
  Objects and Classes -> Code Runners [olive] -> Unit Testing [olive] -> Java Proficiency,
  Compiling and Running [red] -> Sequencing,
  File Systems -> Tables -> Relational Data  -> Joins -> Relational Queries -> Data Literacy,
  Proof by Substitution -> Proof by Induction -> Math Literacy,
  Broadcast Networking -> Certificate Encryption -> Cyber Literacy,
};
\newpage
\section*{Second Year}\hspace{-5em}
\tikz \graph [layered layout, nodes={draw,rounded corners}, level distance=15mm, grow=up] {
  Java Proficiency[olive] -> { Recursive Data -> {
                                Recursion -> Iteration Recursion Iso -> Invariants -> {
                                    Recurrence Relations -> Divide and Conquer -> {
                                        Sorting -> {Merge Sort, Quick Sort, Radix Sort}, Dynamic Programming -> Advanced Programming [red]
                                    }, 
                                    Iteration Recursion Iso -> Binary Tree,
                                    Binary Tree -> Binary Search Tree,
                                    Binary Tree -> Graphs,
                                    Binary Tree -> {
                                        Binary Search Tree -> {
                                            Priority Queue and Heaps [gray] -> Abstract Data Types [red]
                                        }, 
                                        Binary Tree -> COMP3000 [red],
                                        Graphs [gray] -> {Graph Traversal [gray],
                                              Graph Applications [gray]
                                        } -> COMP3010 [red]
                                        },
                                        Loop Invariants -> Big oh -> Computability [gray] -> COMP3010 [red], 
                                        Invariants 
                                }
                                },
            Tables [gray] -> Hash Tables [gray] -> Hash Functions [gray] -> Collision Resolution [gray] -> HPC [red],
           },
           Math Literacy [olive] -> Invariants,
           Java Proficiency -> Inheritance -> Overriding -> Generics -> Lambdas -> Late Binding -> Actors,
           Late Binding -> Semaphores,
           Lambdas -> Paralellism,
           Java Proficiency -> Pointers and Allocation -> Memory Mapping -> Scheduling,
           Algorithm Comparison [olive] -> Invariants,
           Recursion Intro [olive] -> Recursion,
};
\newpage
\end{document}

